\documentclass{article}
\usepackage[left=2.5cm, right=2.5cm]{geometry}
\usepackage{caption}
\usepackage{subcaption}
\usepackage{pgfplots}
\usepgfplotslibrary{external}
\tikzexternalize[prefix=figurecache/] 
\pgfplotsset{compat = 1.17}


\begin{document}

\begin{figure}
    \centering
    \begin{tikzpicture}%[width=0.8\linwidth]
    \begin{axis}[
        title={One spin, seen from above},
        %black,
        xlabel = $x$,
        ylabel = $y$,
    ]
        \addplot [
            mark = none,
        ] table [
            x = x0,
            y = y0,
        ]{./data/one_spin_alpha_0-0.csv};
        \addlegendentry{Computed}
        \addplot [
            mark = none,
            style = dotted,
        ] table [
            x = analytical_x0,
            y = analytical_y0,
        ]{./data/one_spin_alpha_0-0.csv};
        \addlegendentry{Analytical}
    \end{axis}
\end{tikzpicture}

\end{figure}

\begin{figure}
    \centering
    \begin{tikzpicture}
    \begin{axis}[
        black,
        xlabel = t,
        ylabel = Spin amplitude,
    ]
        \addplot [
            mark = none,
        ] table [
            y = y0,
        ]{./data/one_spin_alpha_0-0.csv};
        \addlegendentry{Computed}
        \addplot [
            mark = none,
            style = dotted,
        ] table [
            y = analytical_y0,
        ]{./data/one_spin_alpha_0-0.csv};
        \addlegendentry{Analytical}
    \end{axis}
\end{tikzpicture}

\end{figure}

\begin{figure}
    \centering
    \begin{tikzpicture}
    \begin{axis}[
        title={Amplitude in $z$-direction},
        black,
        xlabel = $t$,
        ylabel = $z$,
        ymin=0,
        ymax=1.5
    ]
        \addplot [
            mark = none,
        ] table [
            y = z0,
        ]{./data/one_spin_alpha_0-0.csv};
        \addlegendentry{Computed}
        \addplot [
            mark = none,
            style = dotted,
        ] table [
            y = analytical_z0,
        ]{./data/one_spin_alpha_0-0.csv};
        \addlegendentry{Analytical}
    \end{axis}
\end{tikzpicture}

\end{figure}

\begin{figure}[htbp]
    \centering
    \begin{subfigure}[t]{0.4\textwidth}
            \begin{tikzpicture}
\begin{groupplot}[
    group style={
        group name = convergence,
        group size = 2 by 1,
        vertical sep = 50 pt,
        ylabels at=edge left,
        yticklabels at=all,
    },
    xlabel = N,
    ylabel = Numerical error,
    width=0.6\linewidth,
    xmode = log,
    ymode = log,
    yminorticks=true,
    %x dir=reverse,
]
\nextgroupplot
    \addplot [
        mark = square*,
        %style = dotted,
    ] table [
        y = heun,
    ]{./data/errors.csv};
    \addlegendentry{Heun's method}
    \addplot [
        mark = o,
        %style = dotted,
    ] table [
        y = euler,
    ]{./data/errors.csv};
    \addlegendentry{Euler's method }

\nextgroupplot [
    legend style={at={(0.02,0.02)}, anchor=south west},
]
    \addplot [
        mark = square*,
        %style = dotted,
    ] table [
        y = heun,
    ]{./data/errors.csv};
    \addlegendentry{Heun's method}
    \addplot [
        mark = o,
        %style = dotted,
    ] table [
        y = euler,
    ]{./data/errors.csv};
    \addlegendentry{Euler's method}
    \addplot [
        mark = triangle,
        %style = dotted,
    ] table [
        y = rk4,
    ]{./data/errors.csv};
    \addlegendentry{Runge-Kutta 4}
\end{groupplot}
\end{tikzpicture}

        \label{fig:convergence_plot}
        \caption{Convgergence plot showing the difference between Heun's method and Euler's method.}
    \end{subfigure}
    \hfill
    \begin{subfigure}[t]{0.4\textwidth}
            \begin{tikzpicture}
\begin{axis}[
    xlabel = N,
    ylabel = {},
    xmode = log,
    ymode = log,
    yticklabel pos = right,
    x dir=reverse,
    legend style={at={(0.98,0.5)}, anchor=north east}
]
    \addplot [
        mark = square*,
        %style = dotted,
    ] table [
        y = heun,
    ]{./data/errors.csv};
    \addlegendentry{Heun's method}
    \addplot [
        mark = o,
        %style = dotted,
    ] table [
        y = euler,
    ]{./data/errors.csv};
    \addlegendentry{Euler's method}
    \addplot [
        mark = triangle,
        %style = dotted,
    ] table [
        y = rk4,
    ]{./data/errors.csv};
    \addlegendentry{Runge-Kutta 4}
\end{axis}
\end{tikzpicture}

        \label{fig:convergence_plot_all}
        \caption{Convgergence plot in which the fourth order Runge-Kutta method is included in addition to Euler's and Heun's method.}
    \end{subfigure}
    \caption{Convergence plots. The figures clearly show that the classical Runge-Kutta method converges significantly faster than the other two methods. Considering runtimes, using this method with a larger step length is also beneficial. Regarding the two lower-order methods, the measured runtimes may ssuggest that Euler's method is actually more efficient than Heun's method.}
\end{figure}

\end{document}
