\documentclass{beamer}
\usepackage{pgfplots}
\usepgfplotslibrary{groupplots}
\usepgfplotslibrary{external}
\tikzexternalize[prefix=figurecache/] 
\pgfplotsset{compat = 1.17}

\newcommand{\Oh}{\mathcal{O}}

\title{Simulating spin magnons using Runge-Kutta methods}
\author{Jonas Bueie}

\begin{document}
\frame{\titlepage}

%\maketitle
%\begin{multicols}{2}

%\section{Methods}
%We will primarily use  Heun's method for calculations.
%This is a second order Runge-Kutta method, meaning that the error is expected to fall as $\Oh(h^{-2})$ as the spacial step size $h$ is reduced.
%As a comparison, two oher methods are used: Euler's method (first order) and the "classical" Runge-Kutta method, which is of fourth order.
%
%
%\section{Results and discussion}
%In this section, the results are presented in a similar order as the tasks in \cite{tasks}.
%First, a single spin is considered, and an initial error analysis is performed.
%From there, new phenomena, including damping and more particles are gradually introduced, whilst the performance of the method is monitored and discussed.
%Finally, magnons propagation through a chain of spins is presented and discussed.

%\subsection{One spin}

%%%%%%%%%%%%%%%%%%%%%%%
%% Task 2.1: One spin %
%%%%%%%%%%%%%%%%%%%%%%%

\begin{frame}
    \vfill
    \centering
    \usebeamerfont{title}
    Part one: Single spin 
    \vfill
\end{frame}

% 1: Tilt the spin

\begin{frame}
\begin{figure}
    \centering
    \begin{tikzpicture}
    \begin{axis}[
        black,
        xlabel = t,
        ylabel = Spin amplitude,
    ]
        \addplot [
            mark = none,
        ] table [
            y = y0,
        ]{./data/one_spin_alpha_0-0.csv};
        \addlegendentry{Computed}
        \addplot [
            mark = none,
            style = dotted,
        ] table [
            y = analytical_y0,
        ]{./data/one_spin_alpha_0-0.csv};
        \addlegendentry{Analytical}
    \end{axis}
\end{tikzpicture}

    %\caption{Amplitude in $y$-diretion, plotted against time, of the computed and analytical solution for the simplest case of one particle in a magnetic field, with no damping and with an initial tilt. The time of the simulation and the analytical calculation are not properly aligned, because the time must be set by hand in order to achieve such an alignment.}
    %\label{error_y}
\end{figure}
\end{frame}
\begin{frame}
\begin{figure}
    \centering
    \begin{tikzpicture}
    \begin{axis}[
        title={Amplitude in $z$-direction},
        black,
        xlabel = $t$,
        ylabel = $z$,
        ymin=0,
        ymax=1.5
    ]
        \addplot [
            mark = none,
        ] table [
            y = z0,
        ]{./data/one_spin_alpha_0-0.csv};
        \addlegendentry{Computed}
        \addplot [
            mark = none,
            style = dotted,
        ] table [
            y = analytical_z0,
        ]{./data/one_spin_alpha_0-0.csv};
        \addlegendentry{Analytical}
    \end{axis}
\end{tikzpicture}

    %\caption{Amplitude of computed and analytical solution for the simplest case of one particle with no damping.}
    %\label{error_z}
\end{figure}
\end{frame}
\begin{frame}
\begin{figure}
    \centering
    \begin{tikzpicture}%[width=0.8\linwidth]
    \begin{axis}[
        title={One spin, seen from above},
        %black,
        xlabel = $x$,
        ylabel = $y$,
    ]
        \addplot [
            mark = none,
        ] table [
            x = x0,
            y = y0,
        ]{./data/one_spin_alpha_0-0.csv};
        \addlegendentry{Computed}
        \addplot [
            mark = none,
            style = dotted,
        ] table [
            x = analytical_x0,
            y = analytical_y0,
        ]{./data/one_spin_alpha_0-0.csv};
        \addlegendentry{Analytical}
    \end{axis}
\end{tikzpicture}

    %\caption{Amplitude in $x$- and $y$-direction of the computed and the analytical solution for the simplest case of one particle in a magnetic field, with no damping and an initial tilt. The figure shows that the spin is rotating in the $xy$-plane with a constant angle against the $z$-axis, as is expected for this particular case.}
    %\label{error_x}
\end{figure}
\end{frame}

% 2: Error analysis
\begin{frame}
\begin{figure}[htbp]
    \centering
    \begin{tikzpicture}
\begin{axis}[
    xlabel = N,
    ylabel = {},
    xmode = log,
    ymode = log,
    yticklabel pos = right,
    x dir=reverse,
    legend style={at={(0.98,0.5)}, anchor=north east}
]
    \addplot [
        mark = square*,
        %style = dotted,
    ] table [
        y = heun,
    ]{./data/errors.csv};
    \addlegendentry{Heun's method}
    \addplot [
        mark = o,
        %style = dotted,
    ] table [
        y = euler,
    ]{./data/errors.csv};
    \addlegendentry{Euler's method}
    \addplot [
        mark = triangle,
        %style = dotted,
    ] table [
        y = rk4,
    ]{./data/errors.csv};
    \addlegendentry{Runge-Kutta 4}
\end{axis}
\end{tikzpicture}

    %\label{fig:convergence_plot_all}
    %\caption{Convergence plots. The figures clearly show that the classical Runge-Kutta method converges significantly faster than the other two methods. Considering runtimes, using this method with a larger step length is also beneficial. Regarding the two lower-order methods, the measured runtimes may ssuggest that Euler's method is actually more efficient than Heun's method.}
\end{figure}
\end{frame}

% 3: Damping
% TODO: Analyze exponential behaviour.
\begin{frame}
\begin{figure}
    \centering
    \begin{tikzpicture}%[width=0.8\linewidth]
    \begin{axis}[
        title={One spin, $\alpha = 0.05$},
        xlabel = $x$,
        ylabel = $y$,
        legend pos=south east,
    ]
        \addplot [
            color = black,
        ] table [
            mark=none,
            x = x0,
            y = y0,
        ]{./data/one_spin_heun_step_2999_alpha_0-05.csv};
        \addlegendentry{Damped}
        \addplot [
            style=dashed,
            color = black,
        ] table [
            mark=none,
            x = analytical_x0,
            y = analytical_y0,
        ]{./data/one_spin_alpha_0-0.csv};
        \addlegendentry{Not damped}
    \end{axis}
\end{tikzpicture}

\end{figure}
\end{frame}

\begin{frame}
\begin{figure}
    \centering
    \begin{tikzpicture}%[width=0.8\linewidth]
    \begin{axis}[
        title={One spin, $\alpha = 0.05, \, \omega = 30/13 \approx 2.3$},
        xlabel = $t$,
        ylabel = {Amplitude ($x$)},
        legend pos=south east,
    ]
        \addplot [
            color = black,
        ] table [
            mark=none,
            x = t,
            y expr = {\thisrow{x0}},
        ]{./data/one_spin_heun_step_2999_alpha_0-05.csv};
        \addlegendentry{Damped}
        \addplot [
            style=dashed,
            color = black,
        ] table [
            mark=none,
            y expr = {0.1*exp(-\thisrow{t}*(0.05*30/13)},
        ]{./data/one_spin_heun_step_2999_alpha_0-05.csv};
        \addlegendentry{$\exp(-t/\tau)$}
    \end{axis}
\end{tikzpicture}

\end{figure}
\end{frame}

\begin{frame}
\begin{figure}
    \centering
    \begin{tikzpicture}%[width=0.8\linewidth]
    \centering
    \begin{groupplot}[
    group style={
        group name = task_4,
        group size = 2 by 1,
        horizontal sep=5pt,
        ylabels at=edge left,
        yticklabels at=edge left,
    },
    width=0.58\linewidth,
    height=0.58\linewidth,
    separate axis lines,
    xlabel={$t$},
    ylabel={Amplitude ($x$)},
]
    \nextgroupplot[
        title = {$\alpha = 0.1$}
    ]
        \addplot [
            color = black,
        ] table [
            mark=none,
            x = t,
            y = x0,
        ]{./data/one_spin_damping_heun_step_2999_alpha_0-1.csv};
        \addlegendentry{Damped}
        \addplot [
            style=dashed,
            color = black,
        ] table [
            mark=none,
            y expr = {0.1*exp(-\thisrow{t}*(0.1*30/13)},
        ]{./data/one_spin_damping_heun_step_2999_alpha_0-1.csv};
        \addlegendentry{$\exp(-t/\tau)$}
    \nextgroupplot[
        title = {$\alpha = 0.2$}
    ]
        \addplot [
            color = black,
        ] table [
            mark=none,
            x = t,
            y = x0,
        ]{./data/one_spin_damping_heun_step_2999_alpha_0-2.csv};
        \addlegendentry{Damped}
        \addplot [
            style=dashed,
            color = black,
        ] table [
            mark=none,
            y expr = {0.1*exp(-\thisrow{t}*(0.2*30/13)},
        ]{./data/one_spin_damping_heun_step_2999_alpha_0-2.csv};
        \addlegendentry{$\exp(-t/\tau)$}
    \end{groupplot}
    \begin{axis}[
        %title={Varying the damping},
        %xlabel = $t$,
        %ylabel = {Amplitude ($x$)},
        hide x axis,
        hide y axis,
    ]
    \end{axis}
\end{tikzpicture}

\end{figure}
\end{frame}


%%%%%%%%%%%%%%%%%%%%%%%%%%
%% Task 2.2: Spin chain %%
%%%%%%%%%%%%%%%%%%%%%%%%%%

\begin{frame}
    \vfill
    \centering
    \usebeamerfont{title}
    Part two: Spin chain
    \vfill
\end{frame}

% 2.2.1: Ground states. Random initial directions.

\begin{frame}
\begin{figure}
    \centering
    \begin{tikzpicture}
    \begin{axis}[
        title={Random initial directions, $J = +1$},
        xlabel = $t$,
        ylabel = $z$,
        skip coords between index={7000}{10000},
    ]
        \addplot table [
            mark=none,
            x = t,
            y = z0,
        ]{./data/10_spins_random_ferro_alpha_0-05.csv};
        \addplot table [
            mark=none,
            y = z1,
        ]{./data/10_spins_random_ferro_alpha_0-05.csv};
        \addplot table [
            mark=none,
            y = z2,
        ]{./data/10_spins_random_ferro_alpha_0-05.csv};
        \addplot table [
            mark=none,
            y = z3,
        ]{./data/10_spins_random_ferro_alpha_0-05.csv};
        \addplot table [
            mark=none,
            y = z4,
        ]{./data/10_spins_random_ferro_alpha_0-05.csv};
        \addplot table [
            mark=none,
            y = z5,
        ]{./data/10_spins_random_ferro_alpha_0-05.csv};
        \addplot table [
            mark=none,
            y = z6,
        ]{./data/10_spins_random_ferro_alpha_0-05.csv};
        \addplot table [
            mark=none,
            y = z7,
        ]{./data/10_spins_random_ferro_alpha_0-05.csv};
        \addplot table [
            mark=none,
            y = z7,
        ]{./data/10_spins_random_ferro_alpha_0-05.csv};
        \addplot table [
            mark=none,
            y = z8,
        ]{./data/10_spins_random_ferro_alpha_0-05.csv};
        \addplot table [
            mark=none,
            y = z9,
        ]{./data/10_spins_random_ferro_alpha_0-05.csv};
    \end{axis}
\end{tikzpicture}

\end{figure}
\end{frame}

\begin{frame}
\begin{figure}
    \centering
    \begin{tikzpicture}
    \begin{axis}[
        xlabel = $t$,
        ylabel = $z$,
    ]
        \addplot table [
            mark=none,
            x = t,
            y = z0,
        ]{./data/2999_spins_magnon_random_alpha_0-05.csv};
        \addplot table [
            mark=none,
            y = z1,
        ]{./data/2999_spins_magnon_random_alpha_0-05.csv};
        \addplot table [
            mark=none,
            y = z2,
        ]{./data/2999_spins_magnon_random_alpha_0-05.csv};
        \addplot table [
            mark=none,
            y = z3,
        ]{./data/2999_spins_magnon_random_alpha_0-05.csv};
        \addplot table [
            mark=none,
            y = z4,
        ]{./data/2999_spins_magnon_random_alpha_0-05.csv};
        \addplot table [
            mark=none,
            y = z5,
        ]{./data/2999_spins_magnon_random_alpha_0-05.csv};
        %\addplot table [
        %    mark=none,
        %    y = z6,
        %]{./data/2999_spins_magnon_random_alpha_0-05.csv};
        %\addplot table [
        %    mark=none,
        %    y = z7,
        %]{./data/2999_spins_magnon_random_alpha_0-05.csv};
        %\addplot table [
        %    mark=none,
        %    y = z7,
        %]{./data/2999_spins_magnon_random_alpha_0-05.csv};
        %\addplot table [
        %    mark=none,
        %    y = z8,
        %]{./data/2999_spins_magnon_random_alpha_0-05.csv};
        %\addplot table [
        %    mark=none,
        %    y = z9,
        %]{./data/2999_spins_magnon_random_alpha_0-05.csv};
    \end{axis}
\end{tikzpicture}

\end{figure}
\end{frame}

% 2.2.2: Magnons. 

% 1: No coupling. All spins rotate independently.
\begin{frame}
\begin{figure}
    \centering
    \begin{tikzpicture}
    \begin{axis}[
        title={10 spins with random directions. $\alpha = 0, \, J = 0$},
        xlabel = $x $,
        ylabel = $y$,
    ]
        \addplot[
            %color=black,
            style=solid,
		] table [
            mark=none,
            x = x0,
            y = y0,
        ]{./data/10_spins_random_no_coupling_alpha_0-0.csv};
        \addplot[
            %color=black,
            style=solid,
		] table [
            mark=none,
            x = x1,
            y = y1,
        ]{./data/10_spins_random_no_coupling_alpha_0-0.csv};
        \addplot[
            %color=black,
            style=solid,
		] table [
            mark=none,
            x = x2,
            y = y2,
        ]{./data/10_spins_random_no_coupling_alpha_0-0.csv};
        \addplot[
            %color=black,
            style=solid,
		] table [
            mark=none,
            x = x3,
            y = y3,
        ]{./data/10_spins_random_no_coupling_alpha_0-0.csv};
        \addplot[
            %color=black,
            style=solid,
		] table [
            mark=none,
            x = x4,
            y = y4,
        ]{./data/10_spins_random_no_coupling_alpha_0-0.csv};
        \addplot[
            %color=black,
            style=solid,
		] table [
            mark=none,
            x = x5,
            y = y5,
        ]{./data/10_spins_random_no_coupling_alpha_0-0.csv};
        \addplot[
            %color=black,
            style=solid,
		] table [
            mark=none,
            x = x6,
            y = y6,
        ]{./data/10_spins_random_no_coupling_alpha_0-0.csv};
        \addplot[
            %color=black,
            style=solid,
		] table [
            mark=none,
            x = x7,
            y = y7,
        ]{./data/10_spins_random_no_coupling_alpha_0-0.csv};
        \addplot[
            %color=black,
            style=solid,
		] table [
            mark=none,
            x = x8,
            y = y8,
        ]{./data/10_spins_random_no_coupling_alpha_0-0.csv};
        \addplot[
            %color=black,
            style=solid,
		] table [
            mark=none,
            x = x9,
            y = y9,
        ]{./data/10_spins_random_no_coupling_alpha_0-0.csv};
    \end{axis}
\end{tikzpicture}

\end{figure}
\end{frame}

% 2: Ferromagnetic coupling. A magnon propagating.
\begin{frame}
\begin{figure}
    \centering
    \begin{tikzpicture}
    \begin{axis}[
        title = {Magnon, $J = +1, \, \alpha = 0$},
        xlabel = $t$,
        ylabel = $z$,
        skip coords between index={2000}{10000},
    ]
        \addplot table [
            mark=none,
            x = t,
            y = z0,
        ]{./data/10_spins_magnon_ferro_alpha_0-0.csv};
        \addplot table [
            mark=none,
            y = z1,
        ]{./data/10_spins_magnon_ferro_alpha_0-0.csv};
        \addplot table [
            mark=none,
            y = z2,
        ]{./data/10_spins_magnon_ferro_alpha_0-0.csv};
        \addplot table [
            mark=none,
            y = z3,
        ]{./data/10_spins_magnon_ferro_alpha_0-0.csv};
        \addplot table [
            mark=none,
            y = z4,
        ]{./data/10_spins_magnon_ferro_alpha_0-0.csv};
        \addplot table [
            mark=none,
            y = z5,
        ]{./data/10_spins_magnon_ferro_alpha_0-0.csv};
        \addplot table [
            mark=none,
            y = z6,
        ]{./data/10_spins_magnon_ferro_alpha_0-0.csv};
        \addplot table [
            mark=none,
            y = z7,
        ]{./data/10_spins_magnon_ferro_alpha_0-0.csv};
        \addplot table [
            mark=none,
            y = z7,
        ]{./data/10_spins_magnon_ferro_alpha_0-0.csv};
        \addplot table [
            mark=none,
            y = z8,
        ]{./data/10_spins_magnon_ferro_alpha_0-0.csv};
        \addplot table [
            mark=none,
            y = z9,
        ]{./data/10_spins_magnon_ferro_alpha_0-0.csv};
    \end{axis}
\end{tikzpicture}

\end{figure}
\end{frame}

% 3: Talk a bit about magnons.

% 4: Magnons with damping.
\begin{frame}
\begin{figure}
    \centering
    \begin{tikzpicture}
    \begin{axis}[
        title={Magnon, $J = +1, \, \alpha = 0.05$},
        xlabel = $t$,
        ylabel = $z$,
        skip coords between index={1500}{10000},
    ]
        \addplot table [
            mark=none,
            x = t,
            y = z0,
        ]{./data/10_spins_magnon_ferro_alpha_0-05.csv};
        \addplot table [
            mark=none,
            y = z1,
        ]{./data/10_spins_magnon_ferro_alpha_0-05.csv};
        \addplot table [
            mark=none,
            y = z2,
        ]{./data/10_spins_magnon_ferro_alpha_0-05.csv};
        \addplot table [
            mark=none,
            y = z3,
        ]{./data/10_spins_magnon_ferro_alpha_0-05.csv};
        \addplot table [
            mark=none,
            y = z4,
        ]{./data/10_spins_magnon_ferro_alpha_0-05.csv};
        \addplot table [
            mark=none,
            y = z5,
        ]{./data/10_spins_magnon_ferro_alpha_0-05.csv};
        \addplot table [
            mark=none,
            y = z6,
        ]{./data/10_spins_magnon_ferro_alpha_0-05.csv};
        \addplot table [
            mark=none,
            y = z7,
        ]{./data/10_spins_magnon_ferro_alpha_0-05.csv};
        \addplot table [
            mark=none,
            y = z7,
        ]{./data/10_spins_magnon_ferro_alpha_0-05.csv};
        \addplot table [
            mark=none,
            y = z8,
        ]{./data/10_spins_magnon_ferro_alpha_0-05.csv};
        \addplot table [
            mark=none,
            y = z9,
        ]{./data/10_spins_magnon_ferro_alpha_0-05.csv};
    \end{axis}
\end{tikzpicture}

\end{figure}
\end{frame}

% 5: Magnons with antiferromagnetic coupling.
\begin{frame}
\begin{figure}
    \centering
    \begin{tikzpicture}
    \begin{axis}[
        title = {Magnon, $J = -1, \, \alpha = 0$},
        xlabel = $t$,
        ylabel = $z$,
    ]
        \addplot table [
            mark=none,
            x = t,
            y = z0,
        ]{./data/10_spins_magnon_antiferro_alpha_0-0.csv};
        \addplot table [
            mark=none,
            y = z1,
        ]{./data/10_spins_magnon_antiferro_alpha_0-0.csv};
        \addplot table [
            mark=none,
            y = z2,
        ]{./data/10_spins_magnon_antiferro_alpha_0-0.csv};
        \addplot table [
            mark=none,
            y = z3,
        ]{./data/10_spins_magnon_antiferro_alpha_0-0.csv};
        \addplot table [
            mark=none,
            y = z4,
        ]{./data/10_spins_magnon_antiferro_alpha_0-0.csv};
        \addplot table [
            mark=none,
            y = z5,
        ]{./data/10_spins_magnon_antiferro_alpha_0-0.csv};
        \addplot table [
            mark=none,
            y = z6,
        ]{./data/10_spins_magnon_antiferro_alpha_0-0.csv};
        \addplot table [
            mark=none,
            y = z7,
        ]{./data/10_spins_magnon_antiferro_alpha_0-0.csv};
        \addplot table [
            mark=none,
            y = z7,
        ]{./data/10_spins_magnon_antiferro_alpha_0-0.csv};
        \addplot table [
            mark=none,
            y = z8,
        ]{./data/10_spins_magnon_antiferro_alpha_0-0.csv};
        \addplot table [
            mark=none,
            y = z9,
        ]{./data/10_spins_magnon_antiferro_alpha_0-0.csv};
    \end{axis}
\end{tikzpicture}

\end{figure}
\end{frame}

\begin{frame}
\begin{figure}
    \centering
    \begin{tikzpicture}
    \begin{axis}[
        title = {Magnon, $J = -1, \, \alpha = 0.05$},
        xlabel = $t$,
        ylabel = $z$,
        skip coords between index={4000}{20000},
    ]
        \addplot table [
            mark=none,
            x = t,
            y = z0,
        ]{./data/10_spins_magnon_antiferro_alpha_0-05.csv};
        \addplot table [
            mark=none,
            y = z1,
        ]{./data/10_spins_magnon_antiferro_alpha_0-05.csv};
        \addplot table [
            mark=none,
            y = z2,
        ]{./data/10_spins_magnon_antiferro_alpha_0-05.csv};
        \addplot table [
            mark=none,
            y = z3,
        ]{./data/10_spins_magnon_antiferro_alpha_0-05.csv};
        \addplot table [
            mark=none,
            y = z4,
        ]{./data/10_spins_magnon_antiferro_alpha_0-05.csv};
        \addplot table [
            mark=none,
            y = z5,
        ]{./data/10_spins_magnon_antiferro_alpha_0-05.csv};
        \addplot table [
            mark=none,
            y = z6,
        ]{./data/10_spins_magnon_antiferro_alpha_0-05.csv};
        \addplot table [
            mark=none,
            y = z7,
        ]{./data/10_spins_magnon_antiferro_alpha_0-05.csv};
        \addplot table [
            mark=none,
            y = z7,
        ]{./data/10_spins_magnon_antiferro_alpha_0-05.csv};
        \addplot table [
            mark=none,
            y = z8,
        ]{./data/10_spins_magnon_antiferro_alpha_0-05.csv};
        \addplot table [
            mark=none,
            y = z9,
        ]{./data/10_spins_magnon_antiferro_alpha_0-05.csv};
    \end{axis}
\end{tikzpicture}

\end{figure}
\end{frame}

% 6: Magnetization.

%\begin{frame}
%\begin{thebibliography}{5}
%    \bibitem{tasks}
%        Tor Nordam. \textit{Exercise 2, 2021 TFY4235 Computational Physics}
%\end{thebibliography}
%\end{frame}

\end{document}

